%%%%%%%%%%%%%%%%%%%%%%%%%%%%%%%%%%%%%%%%%%%%%%%%%%%%%%%%%%%%%%%%%%%%%%%%
%                                                                      %
%     File: Thesis_Background.tex                                      %
%     Tex Master: Thesis.tex                                           %
%                                                                      %
%     Author: Andre C. Marta                                           %
%     Last modified :  2 Jul 2015                                      %
%                                                                      %
%%%%%%%%%%%%%%%%%%%%%%%%%%%%%%%%%%%%%%%%%%%%%%%%%%%%%%%%%%%%%%%%%%%%%%%%

\chapter{State of the Art}
\label{chapter:stateoftheart}




High-Performance Computing - HPC

O tipo de workload a ser executado no computador vai fazer 


%%%%%%%%%%%%%%%%%%%%%%%%%%%%%%%%%%%%%%%%%%%%%%%%%%%%%%%%%%%%%%%%%%%%%%%%
\section{General Purpose Computing on GPUs}
\label{section:gpuarch}

A GPU is a highly parallel programmable processor, that is built to perform the same instruction on a set of data, belonging to the category of processors of \textit{Single Instruction Multiple Data} - SIMD or SIMT? . GPUs were first developed with the processing of computer graphics and images in mind, being able to compute all the pixels that compose an image, simultaneously. 

With the increase in performance of these devices and the appearance of new types of computation workloads, such as computer simulations, machine learning, and data analysis applications that could benefit from the parallel nature of GPUs, these started to be used in general-purpose computing, coining the term GPGPU - General-Purpose Graphical Processing Unit.

\subsection{General Overview of GPU Architecture}

The architecture of the GPU can be roughly divided into the Streaming Multiprocessors and on the Memory.  
\subsubsection{Streaming Multiprocessor}
\subsubsection{Memory }

Falar que devido a ser a placa da AMD que vou falar sobre a arquitetura da AMD

\subsection{AMD GNC Architecture}

Falar do ROCm-smi para mudar a frequencia e tensao da GPU core

Falar dos performance counters

Falar como follow up da AMD RDNA Architecture
%%%%%%%%%%%%%%%%%%%%%%%%%%%%%%%%%%%%%%%%%%%%%%%%%%%%%%%%%%%%%%%%%%%%%%%%
\section{Power Savings Techniques}
\label{section:overview}

\subsection{Architectural techniques for saving energy}
Fazer um diagrama todo bonito a mostrar o sistema de realimentação
\subsection{Operating Points}
\label{section:dvfs}
apresentar no final DVFS 

%%%%%%%%%%%%%%%%%%%%%%%%%%%%%%%%%%%%%%%%%%%%%%%%%%%%%%%%%%%%%%%%%%%%%%%%
\section{Dynamic Control of Voltage and Frequency}
\subsection{Control Theory}
\subsection{GPU DVFS}
\subsubsection{Dynamic Voltage and Frequency Scaling Effects}
\label{section:solarch}
https://www.researchgate.net/publication/261725696_A_Survey_of_Methods_For_Analyzing_and_Improving_GPU_Energy_Efficiency
%%%%%%%%%%%%%%%%%%%%%%%%%%%%%%%%%%%%%%%%%%%%%%%%%%%%%%%%%%%%%%%%%%%%%%%%
\section{Models}
\subsection{Power Modeling}
\label{section:powermodels}
\subsubsection{Empirical Methods}
\subsubsection{Statistical Methods}

%%%%%%%%%%%%%%%%%%%%%%%%%%%%%%%%%%%%%%%%%%%%%%%%%%%%%%%%%%%%%%%%%%%%%%%%
\subsection{Performance Modeling}
\label{section:powermodels}
\subsubsection{Pipeline Analysis}
\subsubsection{Statistical Methods}
Performance Counters

