%%%%%%%%%%%%%%%%%%%%%%%%%%%%%%%%%%%%%%%%%%%%%%%%%%%%%%%%%%%%%%%%%%%%%%%%
%                                                                      %
%     File: Thesis_Introduction.tex                                    %
%     Tex Master: Thesis.tex                                           %
%                                                                      %
%     Author: Andre C. Marta                                           %
%     Last modified :  2 Jul 2015                                      %
%                                                                      %
%%%%%%%%%%%%%%%%%%%%%%%%%%%%%%%%%%%%%%%%%%%%%%%%%%%%%%%%%%%%%%%%%%%%%%%%

\chapter{Introduction}
\label{chapter:introduction}

Insert your chapter material here...

%%%%%%%%%%%%%%%%%%%%%%%%%%%%%%%%%%%%%%%%%%%%%%%%%%%%%%%%%%%%%%%%%%%%%%%%
\section{Motivation}
\label{section:motivation}

Relevance of the subject...
With the advent of Big Data, where a massive amount of data is constantly being collected, a need for incremental increases in the processing power of data centers arises. The term \textit{High-Performance Computing} or HPC refers to the practice of aggregating computing power in a way that delivers much higher performance than one could get out of a typical desktop computer or workstation to solve large problems in science, engineering, or business. [https://insidehpc.com/hpc-basic-training/what-is-hpc/] 

Analyzing the current TOP500 [https://www.top500.org/lists/2019/06/highs/]  supercomputer list as of June 2019, for the first time, all 500 entries have, at least, 1 Petaflop of performance. This is greatly the case due to the use of custom accelerators and GPU in particular. As of the most recent list, 134 systems use accelerators and from those, 127 are equipped with GPUs. However, this massive amount of performance capabilities is tied to the exceptionally high power consumption of the GPUs deployed on this supercomputers, as a reference, the top performer \textit{Summit} supercomputer, with a peak performance of around 200 TFlops/s consumes 10096 kW of power. [https://www.top500.org/lists/2019/06/] Adding to this powering cost, an additional 0.5 to 1W of power is consumed by the cooling system itself [C.D. Patel, C.E. Bash, R. Sharma, M. Beitelmal, and R. Friedrich. 2003. Smart cooling of data centers. Pacific RIM/ASME International Electronics Packaging Technical Conference and Exhibition (IPACK03) (2003).], which makes energy efficiency the most important factor in the creation of these supercomputers.
%%%%%%%%%%%%%%%%%%%%%%%%%%%%%%%%%%%%%%%%%%%%%%%%%%%%%%%%%%%%%%%%%%%%%%%%
\section{Objectives}
\label{section:objectives}

Explicitly state the objectives set to be achieved with this thesis...
