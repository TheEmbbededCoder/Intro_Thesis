%%%%%%%%%%%%%%%%%%%%%%%%%%%%%%%%%%%%%%%%%%%%%%%%%%%%%%%%%%%%%%%%%%%%%%%%
%                                                                      %
%     File: Thesis_Introduction.tex                                    %
%     Tex Master: Thesis.tex                                           %
%                                                                      %
%     Author: Andre C. Marta                                           %
%     Last modified :  2 Jul 2015                                      %
%                                                                      %
%%%%%%%%%%%%%%%%%%%%%%%%%%%%%%%%%%%%%%%%%%%%%%%%%%%%%%%%%%%%%%%%%%%%%%%%

\chapter{Introduction}
\label{chapter:introduction}

With the advent of Big Data, a massive amount of information is constantly being collected and analyzed to extract value out of it. The way to obtain the value is through the use of machine learning algorithms, capable of analyzing huge data sets and uncover relations and extract meaning out of the raw information. These algorithms are being run in all kinds of devices, from supercomputers to our smartphones and a tonic exists across all kinds of modern devices, they are Heterogeneous Computing Systems composed of traditional Central Processing Units (CPUs) and Graphical Processing Units (GPUs). The use of GPUs, with their highly parallel architecture, is bringing significant gains in the performance to the systems and the possibility of increased capabilities of the algorithms being run. 

The addition of this new processing device to the computers also brought the drawback added power consumption. However, unlike CPUs, where it is already common to see advance power techniques that control the frequency and voltage applied to the processor accordingly to the workload, on GPUs the dynamic scaling  of its frequency and voltage (Dynamic voltage and frequency scaling - DVFS) is still mostly relying on external factors such as the temperature of the die and the power that the system is requiring to the power supply. The use of GPUs in such different types of applications, from video games and rendering to scientific simulations and machine learning applications, makes it hard for manufacturers to optimize the DVFS parameters to couple with the different workloads.

In the case of machine learning algorithms, big efforts are being done to enable them to be run with minimum energy consumption. From optimizing specific parts of the GPUs architecture to achieve higher performance to the scheduling of the different kernels being run to maximize the performance and minimize the Dark Silicon phenomenon [Esmaeilzadeh, H., Blem, E., St Amant, R., Sankaralingam, K., and Burger, D. (2011). Dark silicon and the end of multicore scaling. In 38th Annual International Symposium on Computer Architecture, (ISCA’2011)., pages 365–376. IEEE.]. However, if one looks to the nature of machine learning algorithms, the train of it is based on iterative and convergence processes, where small imprecisions of the computation, still leads to correct outputs. These types of applications are called Imprecise Tolerant (IT) applications and their hidden property of tolerance against imprecision can be exploited to increase the efficiency of the circuits running them. In this case, two types of approaches can be made: the creation of custom computational blocks that natively have sources of control imprecision intending to reduce the power consumption [Efficient utilization of imprecise computational blocks for hardware implementation of imprecision tolerant application] or, try to use regular GPUs outside of the circumscribed voltage and frequency margins defined by the GPU manufacturer. While the first approach will require the addition of more hardware to the devices and contribute to the Dark Silicon phenomenon, the second has the great benefit of allowing improvements of the current hardware in the market. 

To verify the feasibility of enabling a degree of imprecise computation in a regular GPU, a convolution neural network (a type of machine learning algorithm specialized for image processing) is trained to identify handwritten numbers of the MNIST data set. This neural network is trained on a heterogeneous computer equipped with an AMD GPU, first with the conventional DVFS parameters and then outside of it. It is verified that it is possible to under-voltage the GPU core 170mV from the default1150mV without any degradation on both the performance of the GPU (required time to train the model) and the achieved accuracy at the end of the training session. More importantly, this reduction on the supplied voltage led to a reduction of 40.48\% of the maximum required power,  28.81\% reduction on the average power and a 26.92\% reduction of the total required energy to train the model.

This result shows that there is space to be explored outside of the conventional DVFS parameters when Imprecise Tolerant applications are being executed.


%%%%%%%%%%%%%%%%%%%%%%%%%%%%%%%%%%%%%%%%%%%%%%%%%%%%%%%%%%%%%%%%%%%%%%%%
\section{Main Objectives}
\label{section:objectives}




This result shows that there is space to be explored on the creation of a control mechanism for the frequency and voltage of a GPU core that is aware of the nature of the application being run and the results that are being produced by it to dramatically reduce the power consumption of this systems.

The main objective of this dissertation is to design a dynamic controller the voltage and frequency of the GPU core that places those parameters outside of the default working point to reduce the power consumption without degradation of the results produced by the system. This controller needs to be aware of the nature of the application being run and the results that it is producing, not only the extrinsic parameters of the GPU like total power consumption and temperature.

